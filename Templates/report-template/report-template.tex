\documentclass[11pt]{article}
%\usepackage[utf8]{inputenc}
\usepackage[margin=1in]{geometry}
\usepackage{graphicx}% include images/graphics
\usepackage{xcolor}% select colors
\setlength{\parindent}{0pt}% no indents
\usepackage[sorting=none]{biblatex}% manage bibliography: do not sort references
\addbibresource{references.bib}% import the bibliography from the same directory

% QUESTIONS | SOLUTION ENVIRONMENTS
% question environment for consistency and to save on typing
\newenvironment{question}
  {\itshape\list{}{\leftmargin=0em\rightmargin=0em}%
   \item\relax}
  {\endlist\ignorespacesafterend}

% solution environment for consistency and to save on typing
\newenvironment{solution}
  {\textbf{Solution: }\list{}{\leftmargin=1em\rightmargin=0em}% <-- \leftmargin=0em to suppress margin
   \item\relax}
  {\endlist\ignorespacesafterend}


\begin{document}

\section*{\centering Written Report – 6.419x Module 1}
\subsection*{\raggedleft Name: (your user name)}


\subsection*{Problem 1.1}
\begin{question}
1. (2 points) 
How would you run a randomized controlled double-blind experiment to determine the effectiveness of the vaccine? Write down procedures for the experimenter to follow. 
(Maximum 200 words)
\end{question}

\begin{solution}
Write your answer in a brief and clear language. In addition, you should add all materials that you have consulted to in the Reference section at the end of the report.  These materials could be a paper \cite{Wasserstein:2016}, a book \cite{Gustavii:2017}, or some internet materials \cite{Wiki:PCA}.
\end{solution}

\begin{question}
2. (3 points) 
For each of the NFIP study, and the Randomized controlled double blind experiment above, which numbers (or estimates) show the effectiveness of the vaccine? Describe whether the estimates suggest the vaccine is effective.
(Maximum 200 words)
\end{question}

\begin{solution}
% write your solution here
\end{solution}


\subsection*{Problem 1.3}
\begin{question}
(a-1). (2 points) 
Your colleague on education studies really cares about what can improve the education outcome in early childhood. He thinks the ideal planning should be to include as much variables as possible and regress children's educational outcome on the set. Then we select the variables that are shown to be statistically significant and inform the policy makers. Is this approach likely to produce the intended good policies?
\end{question}

\begin{solution}
% write your solution here
\end{solution}

\begin{question}
(a-2). (3 points) 
Your friend hears your point, and think it makes sense. He also hears about that with more data, relations are less likely to be observed just by chance, and inference becomes more accurate. He asks, if he gets more and more data, will the procedure he proposes find the true effects?
\end{question}

\begin{solution}
% write your solution here
\end{solution}


\printbibliography% print bibliography here

\end{document}